\documentclass{article} % For LaTeX2e
\usepackage{nips15submit_e,times}
\usepackage{hyperref}
\usepackage{url}
\usepackage{multirow}
\usepackage{array}
\newcolumntype{M}[1]{>{\centering\arraybackslash}m{#1}}
%\documentstyle[nips14submit_09,times,art10]{article} % For LaTeX 2.09


\title{CHORDR: Hidden-Markov-Perceptron for Chord Recognition}


\author{
David S.~Hippocampus\thanks{ Use footnote for providing further information
about author (webpage, alternative address)---\emph{not} for acknowledging
funding agencies.} \\
Department of Computer Science\\
Cranberry-Lemon University\\
Pittsburgh, PA 15213 \\
\texttt{hippo@cs.cranberry-lemon.edu} \\
\And
Coauthor \\
Affiliation \\
Address \\
\texttt{email} \\
\AND
Coauthor \\
Affiliation \\
Address \\
\texttt{email} \\
\And
Coauthor \\
Affiliation \\
Address \\
\texttt{email} \\
\And
Coauthor \\
Affiliation \\
Address \\
\texttt{email} \\
(if needed)\\
}

% The \author macro works with any number of authors. There are two commands
% used to separate the names and addresses of multiple authors: \And and \AND.
%
% Using \And between authors leaves it to \LaTeX{} to determine where to break
% the lines. Using \AND forces a linebreak at that point. So, if \LaTeX{}
% puts 3 of 4 authors names on the first line, and the last on the second
% line, try using \AND instead of \And before the third author name.

\newcommand{\fix}{\marginpar{FIX}}
\newcommand{\new}{\marginpar{NEW}}

%\nipsfinalcopy % Uncomment for camera-ready version

\begin{document}


\maketitle

\begin{abstract}
Analysis of harmonic structure in music often starts with labelling every chord in a musical piece. A system for performing chord recognition is very useful for harmonic analysis and for applications such as music search, music similarity identification and music composition. In this report, we present CHORDR, an improved model for automatic chord recognition based on BREVE’s HMPerceptron model [5]. We compare our results validated on a corpus of chorales from JS Bach and various other classical compositions.
\end{abstract}

\section{Introduction}

Our system takes MIDI input re-encoded as a list of events, each consisting of a set of tones. We aim to assign each event with a chord label. A chord is a set of simultaneously played notes and a chord label posses as information the root note and the mode. In our dataset, we use boolean vectors in $\{0,1\}^{12}$ to represent the presence or absence of the twelve distinct pitch classes. The lowest note is encoded using an integer. Relative accent is encoded using an integer of value 1 to 5 where lower number denotes less accented event and higher number denotes more accented event.

We also used chord progression and accents as contextual information. Together, they form the framework for designing and evaluating a set of Boolean basis functions.

We approach the chord recognition problem as a sequential supervised learning (SSL) problem. Sequential learning can be understood as a path-finding problem over a layered-graph. Here, the vertices represent chord labels which are weighted using vertical basis functions, while transitional weights are calculated using horizontal basis functions. With $T$ layers, and $K$ labels, we have a $T \times K$ graph. Finding the optimal path can be solved using Viterbi algorithm, yielding a complexity of $O(TK^2)$. However, our system uses CarpeDiem algorithm, solving the problem in $O(TK\log(k))$.

\section{Model Formulation}

\subsection{A critique of BREVE's Feature Functions}

In the following, the implemented set of basis functions are described and distinctions between BREVE and ChordRecognition are rationalized. Notably, we denote the current event with xt, and the currently predicted label with yt.

In BREVE, five consecutive entries in the feature vector are used to indicate exactly how many notes of yt are present in xt. Unfortunately, some chord structures contain 3 distinct notes, while others, like the dominant seventh chord, contain 4 distinct notes. Of the supported chord types, approximately half of the chord structures are comprised of 4 distinct notes, however, these chords are used less frequently. As a result, the learned weights prioritize having three notes of yt present in xt, which diminishes the discriminative power of this feature. Consider an event xt, which contains the pitch classes C, E, G, and Bb. Here, xt is a fully stated dominant seventh chord, yet it contains 3 distinct notes of $A_{m7}$, $E_{m6}$, $C_{M}$, $C_{M4}$, $C_{M6}$, $C_{M7}$, $G_{d}$, while containing 4 distinct notes of $C_{V}$. In general this feature aims to gauge the similarity between a collection of notes and a chord label, however this approach is not optimal, as it reports $C_V$ to be less likely than a number of other labels.

Alternatively, five consecutive entries in the feature vector are used to indicate the percentage of notes in xt which are contained in yt. Notably, the percentage is quantized into partitions each with a width of 20\%. Using this approach, the learnt weights reflect the fact that having $(80\%, 100\%]$ is the most desirable condition for a predicted label. As a result, both training error and testing error decreased using this metric.

Furthermore, the model adopted in BREVE makes the assumption that the presence of each note within a given chord structure is equally significant, when in fact this is not the case. For example, given a $C_{M7}$ chord, or any seventh chord for that matter, the presence or absence of a fifth is a very weak indicator, as it is quite common to voice seventh chords without the fifth. In contrast, the presence or absence of a seventh is extremely important, as this chord type specifically dictates its presence. With this in mind, it is somewhat surprising that Radicioni et al. have two features which indicate if the root and added note of a chord yt are present in xt, but neglect to observe the presence of a third or fifth. Additionally, it seems counterintuitive that BREVE only observes if the root of yt is present in xt+1, neglecting to consider the content of xt-1. In light of this reality, ChordRecognition reports the presence of a third and fifth.

The most prominent difficulty in discerning the correct chord label is in cases where xt is contains a small number of distinct pitches. Without contextual information, it is reasonable to assert that an event containing the pitch classes C and G may represent a $C_{m}$, $C_{M}$, $G_{M4}$ or $Ab_{M7}$ chord. Quite frequently, a fully voiced chord is not sounded simultaneously, as the remaining chord tones have been stated previously or are to be stated on the following beat. In order to mitigate these uncertain circumstances, contextual analysis is necessary.

\subsection{Determining the Relevance of CHORDR}

Unfortunately, it is not admissible to take adjacent events into consideration for every input xt, as the nature of adjacent events is only illuminating in specific circumstances. As a result, one experimented with several differing methods for determining a relevant set of events, before determining that the joint criterion of event similarity and accent trajectory was sufficient.

In order to evaluate the similarity between two events xA and xB, the number of common tones, or equivalently the number of elements in xA ∩ xB, are counted. When this value surpasses a threshold, the events are considered to be similar. In most musical genres, chord changes are accented. As a result, a set of unaccented events following an accented event frequently constitutes a single chord structure. Since accents are relative to the surrounding context, a pair of events xA and xA+1 satisfy the accent condition when xA is less accented than xA+1. With regards to the music of J.S. Bach and his Baroque contemporaries, accented beats are directly derived from the metric structure. Accordingly, chord changes often occur on strong beats. Although this musical convention has decreased in prevalence since the Baroque era, chord changes are consistently accented despite the fact that these accents may be syncopated. Additionally, MIDI files contain velocity information which may be used to extrapolate the required accent information. Consequently, this analysis is not limited to the music of J.S. Bach, as it is dependent on accent trajectory not metric structure.

Let xR be a set of consecutive events, necessarily including xt, which collectively satisfy the similarity condition and accent condition. Then, xR is constructed as follows. To begin, xt is the only element contained in xR. Here, xt.accent refers to the relative accent of event t, an integer in [1, 5]. Given L < t, and xL+1 ∈ xR, xL satisfies the accent condition if xL.accent > xL+1.accent and the similarity condition if |xL ∩ xt| > threshold. Given L > t, and xL-1 ∈ xR, xL satisfies the accent condition if xL.accent < xL-1.accent and the similarity condition if |xL ∩ xt| > threshold. To reiterate, xL is an eligible addition to xR if it directly precedes or follows one of the events in xR and satisfies the similarity condition and the accent condition.

\begin{tabular}{ r | c c c c c c c }
  \textbf{Events}        & $x_{28}$ & $x_{29}$ & $x_{30}$ & $x_{31}$ & $x_{32}$ & $x_{33}$ & $x_{34}$ \\
  \textbf{Label}         & $D_{m7}$ & $D_{m7}$ & $D_{m7}$ & $G_M$ & $C_M$ & $F_M$ & $F_M$ \\
  \textbf{Metric Accent} & 5 & 2 & 1 & 3 & 4 & 3 & 2 \\
\end{tabular}

For added clarity, an excerpt from the J.S. Bach dataset will be analyzed, as shown in Table 1. Without considering similarity, $x_R = { x28, x29, x30 }$ with $28 \leq t \leq 30, x_R = \{ x_31 \}$ with $t = 31$, and $x_R = \{ x_{32}, x_{33}, x_{34} \}$ with $32 \leq t \leq 34$. Evidenced by the correct chord labels provided above, this segmentation of events is quite reasonable, however, it is evident that the similarity condition is needed to disassociate $x_{32}$ from $\{ x_{33}, x_{34} \}$.

\subsection{An Overview of Feature Functions in CHORDR}

Upon determining xR, several feature functions make observations on general subsets of xR. Notably, in some cases these subsets may be completely empty, and in that case such features report 0. Let xC be the set difference of xR - xt, such that xC contains all contextually relevant events with the exclusion of xt. Let xS be the set intersection of all events in xR, such that xS only contains pitch classes common to all contextually relevant events.

Here, one will briefly rationalize the importance each aforementioned subset of xR. Clearly, it is important to exclusively consider the contents of xt as it is the event we are interested in labelling. Similarly, it is essential to consider the nature of xC, in accordance with the reasons for contextual analysis discussed above. With regards to xS, quite frequently this set contains very few pitch-classes, however, the presence of even a single pitch class in xS, is significant. Often this pitch class is the root or the fifth of the correct chord label. As a result, it was not beneficial to report if xS is completely stated, as this case is extremely rare and would likely only occur with xR = xt = xS. However, in that case this attribute is already captured by reporting if xt is completely stated.

Notably, accuracy was slightly improved when reporting the asserted-degree of yt on pitch classes shared by xt and xt-1 or xt+1 dependent on xR membership, in contrast to reporting the asserted-degree of yt with respect to each event in xR. This is likely due to the fact that outermost events in xR are more likely to contain notes related to a previous chord structure (i.e suspensions) when xR contains more than 3 events. Since these features are dependent on the presence or absence of a single pitch class, they are quite sensitive to these unrelated pitches. Consequently, it is still relevant to consider ( xR ∩ yt ) / |xR| as the effects of these suspended pitches are significantly lessened. As an added measure, it was found beneficial to record the likelihood of a particular chord type, reported in features [43, 43 + K - 1], where K is number of label types. In the table below, the vertical feature functions are summarized, where highlighted rows signify feature functions developed independently for CHORDR. Notably, the horizontal features in BREVE were not modified.

\begin{table}
  \begin{tabular}{|M{4.5cm}|M{2cm}|M{6cm}|}

    \hline

    \textbf{$\mathbf{\phi}$ class} & \textbf{$\mathbf{\phi}$ number} & \textbf{$\mathbf{\phi}$ description} \\

    \hline

    Penalize Added Note on Weak Beats
    & $1$ & if $X.meter(t) < X.meter(t-1)$ and $y_t$ is an added note chord type $F(1) = 0$ \\

    \hline

    \multirow{4}{*}{Asserted-notes}
    & 2 & root of $y_t$ in $x_t$\\ \cline{2-3}
    & 3 & third of $y_t$ in $x_t$ \\ \cline{2-3}
    & 4 & fifth of $y_t$ in $x_t$ \\ \cline{2-3}
    & 5 & added note of $y_t$ in $x_t$ \\

    \hline

    \multirow{6}{*}{\parbox{4.5cm}{\centering Contextually Relevant Asserted-notes for each Chord Degree \{ root, third, fifth, added note \}}}
    & 6 & root of $y_t$ in $x_t \cap x_{t - 1} given x_{t - 1} in x_{R}$ \\ \cline{2-3}
    & $\dots$ & $\ldots$ \\ \cline{2-3}
    & 10 & root of $y_t$ in $x_t \cap x_{t - 1} given x_{t - 1} in x_{R}$ \\ \cline{2-3}
    & $\ldots$ & $\ldots$ \\ \cline{2-3}
    & 14 & root of $y_t$ in $x_s$ \\ \cline{2-3}
    & $\ldots$ & $\ldots$ \\

    \hline

    \multirow{2}{*}{\parbox{4.5cm}{\centering Root in Context Regardless of Relevance}}
    & 18 & root of $y_t$ in $x_{t - 1}$ \\ \cline{2-3}
    & 19 & root of $y_t$ in $x_{t - 1}$ \\

    \hline

    \multirow{3}{*}{\parbox{4.5cm}{\centering Completely Stated Chords}}
    & 20 & all notes of $y_t$ are in $x_t$ \\ \cline{2-3}
    & 21 & all notes of $y_t$ are in $x_t$ \\ \cline{2-3}
    & 22 & all notes of $y_t$ are in $x_t$ \\ \cline{2-3}

    \hline

    \multirow{4}{*}{\parbox{4.5cm}{\centering Percentage of Notes in $Y_t$ (For every 20\%)}}
    & 23 & $\frac{(x_t \cap y_t)}{|x_t|}$ \\ \cline{2-3}
    & $\ldots$ & $\ldots$ \\ \cline{2-3}
    & 28 & $\frac{(x_R \cap y_t)}{|x_R|}$ \\ \cline{2-3}
    & $\ldots$ & $\ldots$ \\ \cline{2-3}
    & 33 & $\frac{(x_C \cap y_t)}{|x_C|}$ \\ \cline{2-3}
    & $\ldots$ & $\ldots$ \\ \cline{2-3}
    & 38 & $\frac{(x_S \cap y_t)}{|x_S|}$ \\ \cline{2-3}

    \hline

    \multirow{3}{*}{\parbox{4.5cm}{\centering Chord Type Likelihood (K label types)}}
    & 43 & $y_t$ is chord type $M$ \\ \cline{2-3}
    & 44 & $y_t$ is chord type $M4$ \\ \cline{2-3}
    & $\ldots$ & $\ldots$ \\

    \hline

    \multirow{8}{*}{\parbox{4.5cm}{\centering Bass-at-degree}}
    & 43 + K & root of $y_t$ is bass note of $x_t$ \\ \cline{2-3}
    & 44 + K & third of $y_t$ is bass note of $x_t$ \\ \cline{2-3}
    & 45 + K & fifth of $y_t$ is bass note of $x_t$ \\ \cline{2-3}
    & 46 + K & seventh of $y_t$ is bass note of $x_t$ \\ \cline{2-3}
    & 47 + K & root of $y_t$ is bass note of $x_{t+1}$ \\ \cline{2-3}
    & 48 + K & fifth of $y_t$ is bass note of $x_{t+1}$ \\ \cline{2-3}
    & 49 + K & root of $y_t$ is bass note of $x_{t+1}$ \\ \cline{2-3}
    & 50 + K & fifth of $y_t$ is bass note of $x_{t+1}$ \\

    \hline

  \end{tabular}
\end{table}


\begin{table}
  \begin{tabular}{|M{3.5cm}|M{0.25cm}|M{3.5cm}|M{3cm}|M{1cm}|M{3cm}|}
    \hline

    \textbf{$\phi$ class} & & \textbf{$\phi$ description} & \textbf{$\phi$ class} & & \textbf{$\phi$ description} \\

    \hline

    % Row 1

    Penalize Added Note on Weak Beats

    & 1 & if X.meter(t) < X.meter(t-1) and yt is an added note chord type F(1) = 0 &

    \multirow{4}{*}{\parbox{3cm}{\centering Percentage of Notes in Yt (For every 20\%)}}

    & 23 & $( x_t \cap y_t ) / |x_t|$ \\ \cline{1-3}\cline{5-6}

    % Row 2

    \multirow{4}{*}{\parbox{3.5cm}{\centering Asserted-notes}}

    & 2 & root of $y_t$ in $x_t$ &
    & 28 & $( x_R \cap y_t ) / |x_R|$ \\ \cline{2-3}\cline{5-6}

    % Row 3

    & 3 & third of $y_t$ in $x_t$ &
    & 33 & $( x_C \cap y_t ) / |x_C|$ \\ \cline{2-3}\cline{5-6}

    % Row 4

    & 4 & fifth of $y_t$ in $x_t$ &
    & 38 & $( x_S \cap y_t ) / |x_S|$ \\ \cline{2-6}

    % Row 5

    & 5 & added note of $y_t$ in $x_t$ &

    \multirow{3}{*}{\parbox{3cm}{\centering Chord Type Likelihood (K label types)}}

    & 43 & $y_t$ is chord type $M$ \\ \cline{1-3}\cline{5-6}

    % Row 6

    \multirow{6}{*}{\parbox{3.5cm}{\centering Contextually Relevant Asserted-notes for each Chord Degree \{ root, third, fifth, added note \}}}

    & 6 & root of $y_t$ in $x_t$ ∩ $x_{t-1}$ given $x_{t-1}$ in $x_R$ &
    & 44 & $y_t$ is chord type $M4$ \\ \cline{2-3}\cline{5-6}

    % Row 7

    & $\ldots$ & $\ldots$ &
    & $\ldots$ & $\ldots$ \\ \cline{2-6}

    % Row 8

    & 10 & root of $y_t$ in $x_t \cap x_{t+1}$ given $x_{t+1}$ in $x_R$ &

    \multirow{8}{*}{\parbox{3cm}{\centering Bass-at-degree}}

    & $43 + K$ & root of $y_t$ is bass note of $x_t$ \\ \cline{2-3}\cline{5-6}

    % Row 9

    & $\ldots$ & $\ldots$ &
    & $44 + K$ & third of $y_t$ is bass note of $x_t$ \\ \cline{2-3}\cline{5-6}

    % Row 10

    & 14 & root of $y_t$ in $x_S$ &
    & $45 + K$ & fifth of $y_t$ is bass note of $x_t$ \\ \cline{2-3}\cline{5-6}

    % Row 11

    & $\ldots$ & $\ldots$ &
    & $46 + K$ &  seventh of $yt$ is bass note of xt \\ \cline{1-3}\cline{5-6}

    % Row 12

    \multirow{2}{*}{\parbox{3.5cm}{\centering Root in Context Regardless of Relevance}}

    & 18 & root of $y_t$ in $x_{t-1}$  &
    & $47 + K$ & root of $y_t$ is bass note of $x_{t+1}$ \\ \cline{2-3}\cline{5-6}

    % Row 13

    & 19 & fifth of $y_t$ is bass note of $x_{t+1}$  &
    & $48 + K$ & fifth of $y_t$ is bass note of $x_{t+1}$ \\ \cline{1-3}\cline{5-6}

    % Row 14

    \multirow{3}{*}{\parbox{3.5cm}{\centering Completely Stated Chords}}

    & 20 & all notes in $y_t$ are in $x_t$ &
    & $49 + K$ & root of $y_t$ in $x_{t}$ \\ \cline{2-3}\cline{5-6}

    % Row 15

    & 21 & all notes in $y_t$ are in $x_R$ &
    & $50 + K$ & \\ \cline{2-6}

    % Row 16

    & 22 & all notes in $y_t$ are in $x_S$ &
    & $51 + K$ &  \\ \cline{2-3}\cline{5-6}

    \hline

  \end{tabular}
\end{table}

\end{document}
